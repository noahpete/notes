\chapter{Optimization}
\section{Hill Climbing}
The most basic form of optimization we explore is that of the local search.

\begin{definition}[Local Search]
	Search algorithms that maintain a single node and searches by moving to a neighboring node.
\end{definition}

We can implement a local searching using \textbf{hill climbing} in which we do the following:
\begin{itemize}
	\item Check values of neighbors of current best value
	\item If a neighbor has a value closer to the desired value, set that value as current best
\end{itemize}

\begin{algorithm}[H]\label{HillClimb}
	\DontPrintSemicolon
	\caption{HillClimb}
	\KwData{problem}
	cur = initial state of problem \\
	\Repeat{better neighbor does not exist}{
		neighbor = best val neighbor of cur \\
		\If{neighbor not better than current}{
			\Return cur
		}
		cur = neighbor
	}
\end{algorithm}

The problem with this naive approach (\textbf{steepest ascent}) is that we may get stuck at a local extremum rather than the global extremum. Further, we may encounter a "shoulder", where several neighboring values are the same, and get stuck. \par

\section{Simulated Annealing}

Although there are a variety of hill climbing algorithms (stochastic, first-choice, beam, etc.). They have their flaws but the theme is that we never go from a good value to a worse value. Thus, we must overcome local extrema. We can tackle this with \textbf{simulated annealing}. 

\begin{definition}[Simulated Annealing]
	Simulated annealing is akin to a "cooling" physical process.
	\begin{itemize}
		\item Early on, higher "temperature": more likely to accept neighbors that are worse than current state
		\item Later on, lower "temperature": less likely to accept neighbors that are worse than current state
	\end{itemize}
\end{definition}

\begin{algorithm}[H]
	\DontPrintSemicolon
	\caption{SimulatedAnnealing}
	\KwData{problem, max}
	cur = initial state of problem \\
	\For{\(t=1\) to max}{
		\(T=\) Temperature(\(t\)) \\
		neighbor = random neighbor of cur \\
		\(\Delta E =\) how much better neighbor is than cur \\
		\If{\(\Delta  E > 0\)}{
			cur = neighbor
		}
		with probability \(e^{\Delta E / T}\), set cur = neighbor \\
	}
\end{algorithm}

\section{Linear Programming}
In the context where we are trying to optimize for some function or when we have real values we are often trying to minimize or maximize a cost function given a variety of constraints. This is where \textbf{linear programming} becomes useful.

\begin{definition}[Linear Programming]
	Linear programming problems often entail the following:
	\begin{itemize}
		\item Minimize a cost function \(c_1 x_1 + c_2 x_2 + \ldots + c_n x_n\)
		\item With constraints of form \(a_1 x_1 + a_2 x_2 + \ldots + a_n x_n \leq b\) or of form \(a_1 x_1 + a_2 x_2 + \ldots + a_n x_n = b\)
		\item With bounds for each variable \(l_i \leq x_i \leq u_i\) 
	\end{itemize}
\end{definition}

\begin{problem}
	Two machines \(X_1\) and \(X_2\) which cost \$50/hr and \$80/hr to run, respectively. We have the following constraints:
	\begin{itemize}
		\item \(X_1\) requires 5 units of labor, \(X_2\) requires 2 units of labor per hour; we have total of 20 units of labor to spend; goal is to minimize cost
		\item \(X_1\) produces 10 units of output per hour, \(X_2\) produces 12 units of output per hour; company needs 90 units of output
	\end{itemize}
\end{problem}
\begin{answer}
	We can create a cost function:
	\[
		50x_1 + 80x_2
	\]
	and our constraints:
	\[
		5x_1 + 2x_2 \leq 20
	\]
	\[
		10x_1 + 12x_2 \geq 90
	\]
	or equivalently:
	\[
		-10x_1 + -12x_2 \leq -90
	\]
	and then solve using a standard linear programming technique such as Simplex, Interior-Point, etc.
\end{answer}

\section{Constraint Satisfaction}
Constraint satisfaction problems often have some number of variables that must be optimized, but they are subject to some constraints.

\begin{definition}[Constraint Satisfaction Problem]
	A general constraint satisfaction problem consists of the following:
	\begin{itemize}
		\item Set of variables \(\left\{ x_1, \ldots , x_n \right\} \) 
		\item Set of domains for each variable \(\left\{ D_1, \ldots , D_n \right\} \) 
		\item Set of constraints \(C\) 
	\end{itemize}
\end{definition}

\begin{problem}
	We must schedule exams for classes \({A, B, \ldots , G}\) such that no conflicts arise for the students taking the courses. The four students' schedules are:
	\begin{itemize}
		\item \emph{Student 1}: \({A, B, C}\)
		\item \emph{Student 2}: \({B, D, E}\) 
		\item \emph{Student 3}: \({C, E, F}\) 
		\item \emph{Student 4}: \({E, F, G}\) 
	\end{itemize}
	We can represent this graphically using an undirected graph, where the nodes represent exams and the edges indicate that the two exams \emph{cannot} be scheduled for the same time.
	\begin{figure}[H]
		\centering
		\incfig[0.3]{csgraph}
		\label{fig:csgraph}
	\end{figure}
	Here we define our variables, domains, and constraints:
	\begin{itemize}
		\item Variables: \(\left\{ A,B,C,D,E,F,G \right\} \)
		\item Domains: \(\left\{ {Monday, Tuesday, Wednesday} \right\} \) 
		\item Constraints: \(\left\{ A \neq B, A \neq C, B \neq C, \ldots , E \neq G, F \neq G \right\} \)  
	\end{itemize}
	Our goal is to find an assignment of a day to each of the classes such that we don't have any conflicts between the classes. In other words, we are aiming for \textbf{node consistency}.
	\begin{definition}[Node Consistency]
		When all the values in a variable's domain satisfy the variable's \emph{unary} constraints.
	\end{definition}
	Furthermore, we are seeking \textbf{arc consistency}.
	\begin{definition}
		When all the values in a variable's domain satisfy the variable's binary constraints. More formally: to make \(X\) arc-consistent with respect to \(Y\), remove elements from \(X\)'s domain until every choice for \(X\) has a possible choice for \(Y\). 
	\end{definition}
\end{problem}

\begin{definition}[Unary Contraint]
	A constraint involving a \emph{single} variable. (e.g.\(\left\{ A \neq Wednesday \right\} \))
\end{definition}

\begin{definition}[Binary Constraint]
	A constraint involving \emph{two} variables. (e.g.\(\left\{ A \neq  B \right\} \))
\end{definition}

Note that constraints can come of different forms. \textbf{Hard} constraints are absolute requirements, whereas \textbf{soft} constraints are preferences.

\section{Arc Consistency}
We first aim to define a function that, given some consistency problem \(csp\), can make some variable \(X\) \textbf{arc consistent} with respect to another variable, \(Y\).

\begin{algorithm}[H]
	\DontPrintSemicolon
	\caption{Revise}
	\KwData{csp, X, Y}
	revised = false \\
	\For{x in X.domain}{
		\If{no y in Y.domain satisfies constraint for (X, Y)}{
			delete x from X.domain \\
			revised = true \\
		}
	}
	\Return{revised}
\end{algorithm}

We can enforce arc consistency across an entire consistency problem:

\begin{algorithm}[H]
	\DontPrintSemicolon
	\caption{AC-3}
	\KwData{csp}
	queue = all arcs in csp \\
	\While{queue not empty}{
		(X, Y) = \texttt{Dequeue}(queue) \\
		\If{\texttt{Revise}(csp, X, Y)}{
			\If{size of X.domain == 0}{
				\Return{false}
			}
			\For{each Z in X.neighbors - \{Y\}}{
				\texttt{Enqueue}(queue, (Z, X))
			}
		}
	}
	\Return{true}
\end{algorithm}

\subsection{CSPs as Search Problems}
We can reframe CSPs as \emph{search problems}. We do so by defining a CSP as follows:
\begin{itemize}
	\item initial state: empty assignment (no variables)
	\item actions: add a \(\left\{ variable = value \right\} \) to assignment
	\item transition model: shows how adding an assignment changes the assignment
	\item goal test: check if all variables assigned and constraints all satisfied
	\item path cost function: all paths have same cost
\end{itemize}

\section{Backtracking}
We can find a solution to a CSP by simply applying arbitrary assignments to variables one by one until a constraint is broken. Then, we can just backtrack and try another assignment. We eventually find a solution or check every possible assignment.

\begin{algorithm}[H]
	\DontPrintSemicolon
	\caption{Backtrack}
	\KwData{assignment, csp}
	\If{assignmennt complete}{
		\Return{assignment}
	}
	var = \texttt{SelectUnassignedVar}(assignment, csp) \\
	\For{value in \texttt{DomainValues}(var, assignment, csp)}{
		\If{value consistent with assignment}{
			add var = value to assignment \\
			result = \texttt{Backtrack}(assignment, csp) \\
			\If{result \(\neq \) failure}{
				\Return{result}
			}
		}
		remove var = value from assignment
	}
	\Return{failure}
\end{algorithm}
