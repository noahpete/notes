\chapter{The Kernel Abstraction}
\lecture{2}{}{Threads}
What does an OS do? It creates \textbf{abstractions} to make hardware easier to use. Further, it manages \textbf{shared} hardware resources. Operating systems use a variety of abstractions:

\begin{figure}[H]
	\centering
	\incfig[0.45]{abstraction}
	\label{fig:abstraction}
\end{figure}

Recall the complexity issues we've faced while designing an OS: multiple users, programs, I/O devices, etc. We can manage this complexity via:
\begin{itemize}
	\item \textbf{Divide and conquer} 
	\item \textbf{Modularity and abstraction} 
\end{itemize}

\begin{definition}[Process]
	A \textbf{process} is the OS abstraction for \textbf{execution}. It's also known as a \textbf{job} or \textbf{task}. 
\end{definition}

This is advantageous simply due to the alternative if we were using strictly hardware.

\begin{figure}[H]
	\centering
	\incfig[0.2]{hardwareproc}
	\caption{The interface hardware provides.}
	\label{fig:hardwareproc}
\end{figure}

\begin{figure}[H]
	\centering
	\incfig[0.6]{osproc}
	\caption{The interface that OS provides.}
	\label{fig:osproc}
\end{figure}

A process is simply a \emph{running program} and is named using its \textbf{process ID (PID)}. It contains all the state for a running program.
\begin{definition}[Program]
	Programs are static entities with \emph{potential} for execution.
\end{definition} 

