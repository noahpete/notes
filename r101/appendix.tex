\chapter{Eigenvalues and Eigenvectors}
Consider the \textbf{scalar linear difference equation}
\[
    z_{k+1} = az_k
\]
where \(a, z_0 \in \mathbb{C}\). We compute some steps of the equation:
\begin{align*}
    z_1 &= az_0 \\
    z_2 &= a_{z1} = a^2 z_0 \\
    z_3 &= a_{z2} = a^3 z_0 \\
    &\vdots \\
    z_k &= a^k z_0
\end{align*}

\section{Iterating with Matrices: A Case for Eigenvalues}
We now analyze the matrix versions of the above equations:
\[
    x_{k+1} = Ax_k,
\]
where \(A\) is a \(n \cross n\) real matrix. If we allow entries of \(A\) to be complex and \(x_0 \in \mathbb{C}\) we have:
\[
    z_k = A^k z_0.
\]
Now we shift our attention towards finding conditions on \(A\) such that \(\lVert z_k \rVert \) contracts, blows up, or stays bounded as \(k\) tends to infinity. Further, upon being given \(z_0\) we will detail the evolution of \(z_k\) for \(k>0\) . \par

\todo{finish eigenstuff}