\lecture{21}{28 Mar. 12:00}{Cache Wrap-Up}

\section{Cache Wrap-Up}

\begin{problem}
  The \emph{grinder} app running on LC2K with full data forwarding and all brances predicted not-taken has the following frequencies:
  \[
    \text{45\% R-type, 20\% branches} \: \: \: \text{15\% loads, 20\% stores}  
  \]
  In \emph{grinder}, 40\% of branches are taken and 50\% of LWs are followed by an immediate use. The I-cache has a miss rate of 3\% and the D-cache has a miss rate of 6\% (no overlapping of misses). On a miss, the main memory is accessed and has a latency of 100 ns. The clock frequency is 500 MHz. \\
  \\
  What is the CPI of \emph{grinder} on the LC2K?
\end{problem}
\begin{answer}
  Since the clock frequency is 500 MHz, we have a 2 ns cycle time. So the stalls per cache miss have a length of
  \[
    100 \text{ ns} / 2 \text{ ns} = 50 \text{ cycles}.
  \]
  So the CPI will equal:
  \[
    CPI = 1 + \text{data hazard stalls} + \text{ctrl hazard stalls} + \text{I-Cache stalls} + \text{D-Cache stalls}
  \]
  \[
    = 1 + 0.15 \cdot 0.5 \cdot 1 + 0.2 \cdot 0.4 \cdot 3 + 1 \cdot 0.03 \cdot 50 + 0.35 \cdot 0.06 \cdot 50 = \textbf{3.865}.
  \]
\end{answer}

\begin{problem}
  Say you have the following:
  \begin{itemize}
    \item \emph{program}: generates 2 billion loads, 1 billion stores, each 4 bytes in size
    \item \emph{cache}: a 32-byte block which gets a 95\% hit rate on the program
  \end{itemize}
  How many bytes of memory would be read and written if:
  \begin{enumerate}[label=\alph*]
    \item We had no cache?
    \item We had a write-through cache with a no-write allocate policy?
    \item We had a write-back cache with a write-allocate policy? (Assume 25\% of misses result in dirty eviction)
  \end{enumerate}
\end{problem}
\begin{answer} \:
  \begin{enumerate}[label=\alph*]
    \item \emph{No cache}: All stores go to memory and are 4 bytes each, so \(1 \text{ billion} \cdot 4 \text{ bytes} = 4 \text{ billion bytes}\) for writes. Similarly, there are 4 bytes per read, so there are \(2 \text{ billion} \cdot 4 \text{ bytes} = 8 \text{ billion bytes}\) for reads.
    \item \emph{Write-through, no allocate}: No allocate means we bypass cache on a miss, and write-through means we write to cache and memory on hit. So, again all stores will go to memory and are still 4 bytes each, so there are \textbf{4 billion bytes of writes}. Only loads that miss in the cache go to memory, but they read the full cache block. So there are \(2 \text{ billion} \cdot 0.05 \cdot 32 \text{ bytes} = \textbf{3.2 billion bytes of reads}\).
    \item \emph{Write-back, write-allocate}: In this case, store misses result in a cache block being read, causing \(1 \text{ billion stores} \cdot 0.05 \cdot 32 \text{ bytes} = 1.6 \text{ billion bytes}\) to be read due to store misses. Similarly, \emph{load} misses result in cache block being read, causing an additional \(2 \text{ billion loads} \cdot 0.05 \cdot 32 \text{ bytes} = 3.2 \text{ billion bytes} \) to be read. Thus, there are \textbf{4.8 billion bytes read}. We only write to memory when there are dirty evictions, which can be done by both loads and stores. Since there are \(0.05 \cdot 1 \text{ billion stores} \cdot 32 \text{ bytes} \cdot 25\% = 0.4 \text{ billion bytes} \) of writes due to store misses. There are \(2 \text{ billion loads} \cdot 0.05 \cdot 32 \text{ bytes}  \cdot 25\% = 0.8 \text{ billion bytes} \) of writing due to load misses, resulting in \textbf{1.2 billion bytes of writes} in total.
  \end{enumerate}
\end{answer}

\begin{problem}
  Given a 200 MHz processor with 8 KB instruction and data caches and a with memory access latency of 20 cycles. Both caches are 2-way associative. A program running on this processor has a 95\% icache hit rate and a 90\% dcache hit rate. On average, 30\% of the instructions are loads or stores. The CPI of this system, if caches were ideal would be 1. \\
  \\
  Suppose you have the following two options for the next generation processor, which do you pick?
  \begin{enumerate}
    \item Double the clock frequency \(\to\) assume this will increase your memory latency to 40 cycles, and the base CPI of 1 can still be achieved after this change
    \item Double the size of your caches \(\to\) this will increase the instruction cache hit rate to 98\% and the data cache hit rate to 95\%; assume the hit latency is still 1 cycle
  \end{enumerate}
\end{problem}
\begin{answer}
  See end of lecture.
\end{answer}

\begin{problem}
  If you have a 16 KB cache with 32-byte cache lines that is four-way set-associative on a computer with 32-bit addresses:
  \begin{enumerate}[a.]
    \item How many sets do you have?
    \item How many bits do you need for the Offset? Index? Tag?
  \end{enumerate}
\end{problem}
\begin{answer} \;
  \begin{enumerate}[a.]
    \item
    \begin{align*}
      2^14 \text{ bytes} / 2^5 \text{ bytes per block} = 2^9 \text{ blocks} \\
      2^9 \text{ blocks} / 2^2 \text{ blocks per set} = 2^7 \text{ sets} 
    \end{align*}
    \item
    \begin{align*}
      \log (2^5 \text{ bytes per block}) = 5 \text{ bits for Offset} \\
      \log (2^7 \text{ sets}) = 7 \text{ bits for set Index} \\
      32 - 7 - 5 = 20 \text{ bits for Tag} 
    \end{align*}
  \end{enumerate}
\end{answer}

\begin{problem}
  Describe why we have an Icache and a Dcache on nearly all computers rather than just one unified cache.
\end{problem}
\begin{answer}
  It's advantageous to separate them since we typically just read instructions, while we need to also write to data. Separating them allows for better bandwidth since you are going to different places for different things.
\end{answer}

\begin{problem}
  Describe the primary \textbf{advantage} of \emph{write-back} caches over \emph{write-through} caches.
\end{problem}
\begin{answer}
  Each are better in the following scenarios:
  \begin{enumerate}
    \item \textbf{Write-Back} is better when we are writing to the same place \emph{many, many times}.
    \item \textbf{Write-Through} is better when we are writing to the same place \emph{only once}.
  \end{enumerate}
\end{answer}