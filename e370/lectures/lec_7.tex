\lecture{22}{6 Apr. 12:00}{Page Tables Cont.}
How do we do page table translation? We have two options:
\begin{enumerate}
  \item \textbf{Flat} (single-level): use a single, massive array
  \begin{itemize}
    \item One page table lookup
    \item Always takes up a lot of memory
  \end{itemize}
  \item \textbf{Hierarchical} (multi-level): use pointer page that points to child pages spread throughout memory
  \begin{itemize}
    \item Dynamically adjusts memory usage
    \item Typically uses much less memory than single-level
    \item Two (or more) page table lookups
    \item Used in most modern systems
  \end{itemize}
\end{enumerate}

\section{Translation Look-Aside Buffer (TLB)}
To translate a virtual address into a physical address, we must first access the page table in physical memory; this requires (sometimes several) slow memory lookups. We fix this performance problem by avoiding main memory in the translation from virtual to physical pages.

\begin{definition}[Translation Look-Aside Buffer (TLB)]
  A fast cache dedicated to storing a small subset of valid page table entries.
  \begin{remark}
    These generally have a very low (<1\%) miss rate.
  \end{remark}
\end{definition}
~24:44