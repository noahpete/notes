\chapter{Overview of Database Systems}

A \textbf{database} is a collection of data, typically describing the activities of one or more related organizations. They often contain \emph{entities} and \emph{relationships} between those entities. \par

Further, this text focuses on \textbf{relational database systems (RDBMSs)}, which are the most common.

\section{Advantages of a DBMS}
Using a DBMS over traditional file systems has many advantages:
\begin{itemize}
	\item \textbf{Data independence}: applications shouldn't, ideally, have access to details of data representation and storage
	\item \textbf{Efficient data access}: efficient techniques are used for data storage and retrieval
	\item \textbf{Data integrity and security}: the DBMS can enforce integrity constraints
	\item \textbf{Data administration}: centralization of the administration of data can be advantageous
	\item \textbf{Concurrent access and crash recovery}: concurrent access is possible
\end{itemize}

\section{The Relational Model}
The main idea to this model is a \textbf{relation}, which can be thought of as a set of \textbf{records}.