\lecture{23}{3 Apr. 10:30}{RSA Cryptosystem}
\section{RSA Cryptosystem}

Whereas Diffie-Hellman relies on the hardness of the discrete log problem in a group of \emph{known} order, RSA relies on the hardness of \textbf{factoring and finding roots} in a group of \emph{unknown} order.

\subsection{RSA Math Foundation}
First, recall Euler's totient function.
\begin{definition}[Euler's Totient Function]\label{EulerTotient}
    The totient, \(\Phi (n)\), of a positive integer \(n > 1\) is defined as hee number of positive integers less than \(n\) that are coprime to \(n\). The following table shows some function values. \\
    \begin{center}
        \begin{tabular}[h]{|c|c|c|}
            \hline
            \(n\) & \(\Phi (n)\) & numbers coprime to \(n\) \\
            \hline
            1 & 1 & 1 \\
            2 & 1 & 1 \\
            3 & 2 & 1, 2 \\
            4 & 2 & 1, 3 \\
            5 & 4 & 1, 2, 3, 4 \\
            6 & 2 & 1, 5 \\    
            \hline
        \end{tabular}
    \end{center}
\end{definition}

Let \(N=p\cdot q\) be the product of two (huge) distinct primes. Then
\[
    \mathbb{Z}_N^* = \left\{ a \in \mathbb{Z}_N = { 0, \ldots , N - 1} : \text{gcd($a, N$) = 1}  \right\}. 
\]
We start with \(\mathbb{Z}_N\) and remove all multiples of \(p\) (i.e. \(0, p, 2p, \ldots , (q-1)p\)) and all multtiples of \(q\) (i.e \(0, q, 2q, \ldots , (p-1)q\)) double counted 0. This means that
\[
    \left| \mathbb{Z}_N^* \right| = \Phi (N) = p . q - q - p + 1 = (p-1)(q-1). 
\]

\begin{definition}[Euler's Theorem]
    In any group \(G\), \(\forall a \in G\), \(a^{|G|} = 1 \in G\). Say \(G = \mathbb{Z}_N^*\), therefore we have
    \[
        \forall a \in \mathbb{Z} _N^*, a^{\Phi (N)} = a^{(p-1)(q-1)} = 1 \mod N.
    \]  
\end{definition}

By Euclid's Theorem, we can compute \(A, B \in \mathbb{Z}\) such that
\[
    A.e + B.\Phi (N) = \text{gcd}(e, \Phi (N)) = 1.
\]
\[
    \implies A.e = 1 - B.\Phi (N) = 1 \mod \Phi (N)
\]
We can define \(d=A \mod \Phi (N)\) as the multiplicative inverse of \(e \mod \Phi (N)\):
\begin{align*}
    d = e^{-1} \mod \Phi (N) \\
    d.e = 1 \mod \Phi (N).
\end{align*}

\subsection{RSA Function}
The choice of \(N, e, d\) gives us the RSA function and its inverse.

\begin{definition}[RSA Function]
    For \(N = p . q\) where \(p, q\) are large disttinctt primes, and \(e \in \mathbb{Z} _{\Phi (N)}^*\) with \(d=e^{-1} \mod \Phi (N)\) the RSA function
    \[
        \text{RSA}_{N,e}: \mathbb{Z}_N^* \to \mathbb{Z} _N^* ; \; \text{RSA} _{N,e} (x) \coloneqq x^e \mod N
    \]
    is a bijection. The inverse is \(\text{RSA} _{N,d}(y)=\text{RSA} _{N,e}^{-1}(y)=y^d \mod N\).
\end{definition}
\begin{proof}
    \(\text{RSA} _{N,e}\) maps \(\mathbb{Z} _N^* \to \mathbb{Z} _N^*\). Need to show \(\text{RSA} _{N,d} = \text{RSA}_{N,e}^{-1}\). Why? Let \(y=\text{RSA} _{N,e}(x)=x^e \mod N\). Then we have
    \[
        y^d = (x^e)^d = x^{e.d} = x^{e.d + k. \Phi (N)} = x^1 \mod N,
    \]
    where \(k\) is some number that will make \(e.d + k.\Phi (N) = 1\).
    RSA is an example of a \textbf{trapdoor function}. We can efficienly evaluate \(\text{RSA}_{N,e} \) in the \emph{forward} direction, and given \textbf{trapdoor information}, \(d\), we can efficiently invert.
    \begin{remark}
        What about without \(d\)?
    \end{remark}
\end{proof}

\begin{figure}[H]
    \centering
    \incfig[0.8]{rsatrapdoor}
    \caption{RSA function is a trapdoor function: it is easy to go left to right, but not the other way around.}
    \label{fig:rsatrapdoor}
\end{figure}

\subsection{RSA Key Generation}
We define the RSA key generation process for \(\text{GenRSA} (1^n)\) as follows:

\begin{algorithm}[H]\label{RSAKeyGen}
	\DontPrintSemicolon
	\caption{RSA Key Generation}
	\KwData{\(1^n\) }
    let \(p, q\) be large primes having bit lengths approximately related to \(n\) \\
    let \(N=p.q\)\\
    compute \(\Phi (N) = (p-1)(q-1)\) \\
    choose some \(e > 1\) such that \(\text{gcd} (e, \Phi (N)) = 1\); Euclid also gives us \(d=e^{-1} \mod \Phi (N)\) \\
    \Return{\(pk = (N,e)\) and \(sk=(N,d)\)}.  
\end{algorithm}
\begin{remark}
    Common choices for \(e\):
    \begin{itemize}
        \item random value
        \item \(e=3\) if \(3t(p-1), 3t(q-1)\)
        \item \(e=2^{16} + 1\) (prime) (in binary: \(e=1000\ldots 0001\), so exponentiation is faster)
    \end{itemize}
\end{remark}

\begin{definition}[RSA Hardness Assumption]
    Given a public key \((N,e)\) and a \emph{random} \(y \gets \mathbb{Z}_N^*\), it is hard to find the pre-image \(x=y^d=y^{e^{-1}} \mod N\). So, the assumption is that \(\forall \text{ p.p.t } \mathcal{A}\):
    \[
        \text{Adv}^{\text{RSA} } (\mathcal{A}) = \Pr_{(pk=(N,e), sk) \gets \text{GenRSA}(1^n), y \gets \mathbb{Z}_N^* } \left[ \mathcal{A}(1^n, (N, e), y) \text{ outputs } x=\text{RSA}_{N,e}^{-1}(y) \right] = \text{negl} (n).
    \] 
\end{definition}

How plausible is the \textbf{RSA hardness assumption}? Well, we can note the following:
\begin{enumerate}[label=(\roman*)]
    \item RSA \(\leq \) Factoring; i.e. if there is an efficient algorithm for factoring integers into their prime factors, then there is one for solving RSA.
    \item Factoring \(\equiv\) Finding \(\Phi (N)\) from \(N\).
    \item Factoring \(\equiv \) Finding \(d\) from \((N, e)\).
\end{enumerate}
\begin{remark}
    It is noteworthy, however, that despite RSA \(\leq \) Factoring, it is \emph{unknown} whether RSA \(\equiv \) Factoring. In other words, we know Factoring is at least as hard as RSA, but we don't know if it is \emph{just} as hard.
\end{remark}

\subsection{RSA Encryption}
Say our encryption scheme is the "textbook" RSA encryption scheme with a key generator \(\text{Gen} (1^n): (pk = (N, e), sk = (N, d))\), defined as: 
\begin{itemize}
    \item Enc(\(pk = (N,e), m \in \mathbb{Z} _N^*\)): output \(c = \text{RSA}_{N, e} (m) = m^e \mod N\)
    \item Dec(\(sk = (N, d), c \in \mathbb{Z} _N^*\)): output \(m = \text{RSA} _{N,d} (c) = c^d \mod N\)
\end{itemize}
This scheme is correct, but since \emph{Enc is deterministic}, it is \textbf{not CPA secure}. 