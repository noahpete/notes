\lecture{22}{29 Mar. 10:30}{CPA Model, El Gamal Cryptosystem}

The above implies we can encrypt long messages bit-by-bit (or block-by-block) or broken up any reasonable way. One call to \(Enc\) on "long" messages translates to many calls on "short" messages, which is fine by the theorem above. \par

\begin{theorem}
  Any public key encryption scheme with a \textbf{deterministic} \(Enc_{k_p}(.)\) algorithm can't be CPA secure \emph{even for one query}!
\end{theorem}
\begin{proof}
  Query \(c \gets LR_{k_p, b}(m_0, m_1)\) for any \(m_0 \neq m_1\). Then run \(c'= Enc_{k_p}(m_0)\). If \(c=c'\) output 0, else 1. This strategy has perfect advantage.
\end{proof}

\section{El Gamal Cryptosystem}
We can construct a CPA secure PKE scheme using Diffie-Hellman. We can represent a message as some \(m \in G\). The "one-time-pad effect" would involve \emph{multiplying} \(m\) with something random \((\mathcal{K}) \).

\begin{definition}[El Gamal]
  We have a scheme:
  \begin{itemize}
    \item Gen(\(1^n\)): choose random \(a \gets \mathbb{Z}_q\), output (\(k_p = \mathcal{A} = g^a \in G, k_s=a\))
    \item Enc(\(k_p = \mathcal{A}, m \in G\)): choose random \(b \gets \mathbb{Z}_q\), output ciphertext (\(\mathcal{B} = g^b \in G, \mathcal{C} = m \cdot \mathcal{A}^b \ \in G\)) 
    \item Dec(\(k_s = a, (\mathcal{B}, \mathcal{C})\)): compute \(\mathcal{K} = \mathcal{B}^a\), output \(\mathcal{C} \cdot \mathcal{K}^{-1} \in G\) 
  \end{itemize}
  And it is correct, as \(\forall m \in G\), \(k_p = g^a, k_s = a\) and:
  \[
    Enc(k_p, \mathcal{A}) = (\mathcal{B}=g^b, \mathcal{C} = m \cdot (g^a)^b)
  \]
  \[
    Dec(\mathcal{B}, \mathcal{C})=\mathcal{C} \cdot (\mathcal{B}^a)^{-1} = m \cdot g^{ab} \cdot (g^{ab})^{-1} = m.
  \]
\end{definition}

\begin{theorem}
  If the \hyperref[ddh]{DDH assumption} holds, then El Gamal is CPA-secure.
\end{theorem}
\begin{proof}
  Assume there is a p.p.t. adversary \(\mathcal{A} \) that can break El Gamal, we can use \(\mathcal{A} \) to build a distinguisher against DDH.

  \begin{figure}[H]
    \centering
    \incfig[0.8]{ddhelgam}
    \caption{A distinguisher using an adversary that can break El Gamal can break DDH.}
    \label{fig:ddhelgam}
  \end{figure}

  In the case that \((g, \mathcal{A}, \mathcal{B}, \mathcal{C})\) is a DH tuple (\textbf{"real"} world), \(\mathcal{D} \) perfectly simulates the left CPA world because \(\mathcal{C} = g^{ab}\). In the \textbf{"ideal"} world, \((g, \mathcal{A}, \mathcal{B}, \mathcal{C})\) is random, so \(\mathcal{D} \) perfectly simulates a "hybrid" CPA world where the ciphertext is two independent random group elements. \par
  Similarly to \(\mathcal{D} \) above, we can construct a \(\mathcal{D}'\) where instead \(\mathcal{A}_{ElGam}\) receives \((\mathcal{B}, m_1 . \mathcal{C})\). Thus, by the triangle inequality we have
  \[
    \mathrm{Adv}^{CPA}(\mathcal{A}) \leq \mathrm{Adv}^{DDH}(\mathcal{D}) + \mathrm{Adv}^{DDH}(\mathcal{D}') = \mathrm{negl} (n) + \mathrm{negl} (n) = \mathrm{negl} (n).
  \]
\end{proof}