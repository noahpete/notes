\lecture{20}{22 Mar. 10:30}{Group Theory}

We continue our review of modular arithmetic. Modular addition, subtraction,
and multiplication are all closed in modular arithmetic. We have the property
\(a = b \mod N \iff N | a - b\) and further, if \(a = a^{\prime} \mod N\) and
\(b = b^{\prime} \mod N\), we have:   
\begin{itemize}
	\item \(a + b = a^{\prime} + b^{\prime} \mod N\) 
	\item \(a - b = a^{\prime} - b^{\prime} \mod N\) 
	\item \(a . b = a^{\prime} . b^{\prime} \mod N\) 
\end{itemize}

However, division is not always possible:
\[
	3 \cdot 2 = 15 \cdot 2 \mod 24 \centernot\implies 3 = 15 \mod 24
\]

\begin{definition}[Invertability]
	An integer \(b\) is \textbf{invertible} if \(\exists c \) where
	\(b \cdot c = 1 \mod N\).
\end{definition}

\begin{lemma}
	If \(b \geq 1, N > 1\), \(b\) is invertible mod \(N \iff \text{gcd}(b, N) = 1\).    
\end{lemma}

\begin{proof}
	Let \(b \cdot c = 1 \mod N\).
	\begin{align*}
		\implies b . c - 1 = N . q \\
		\implies b . c - N . q = 1 . \\
	\end{align*}
	Thus, \(\text{gcd}(b, N) = 1\) since gcd is the smallest positive integer                          
	that is expressible in this way.
\end{proof}

Using the \hyperref[eea]{Extended Euclidean Algorithm}, we can compute inverse mod \(N\)
efficiently.

\begin{definition}[Group]
	\((G, \circ)\), where \(\circ : G \times G \to G\) and \(\circ(g, h)\) is
	denoted as \(g \circ h\), is a \textbf{group} if it satisfies the properties of:
	\begin{itemize}
		\item identity: \(\exists e \in G\) such that \(\forall g \in G \colon e \circ g = g \circ e = g\)  
		\item invertability: \(\forall g \in G, \exists g^{-1} \text{ (or $-g$) }\) such that \(g \circ g^{-1} = e\)  
		\item associativity: \(\forall g_1, g_2, g_3 \in G \colon (g_1 \circ g_2) \circ g_3 = g_1 \circ (g_2 \circ g_3) \) 
		\item (abelian groups): \(\forall g, h \in G, g \circ h = h \circ g\) (i.e commutativity) 
	\end{itemize}   
\end{definition}

\begin{eg}
	For \((\mathbb{Z}_N, + \mod N)\) we have:
	\begin{itemize}
		\item identity: \(a + 0 \equiv 0 + a \equiv a \mod N \)
		\item invertability: \(a + (-a) \equiv 0 \mod N\)
		\item associativity: \(a + (b + c) = (a + b) + c \mod N\)  
	\end{itemize}
\end{eg}

\begin{eg}
	For \((\mathbb{Z}_N^*, . \mod N)\) we have:
	\begin{itemize}
		\item identity: \(a . 1 \equiv 1 . a \equiv a \mod N \)
		\item invertability: \(a . (a^{-1}) \equiv 1 \mod N\)
		\item associativity: \(a . (b . c) = (a . b) . c \mod N\)  
	\end{itemize}
\end{eg}

The size of a group, called the \textbf{group order}, is denoted as
\(|G|\). For example, (\(\mathbb{Z}_N, +\)) has order \(N\), and
\(\mathbb{Z}_p^*, .\) has order \(p - 1\)

\begin{theorem}
	For a group, \(G\), where \(m = |G|\), we have that \(\forall g \in G \colon
	g^m = 1\) (where \(g^m = (((g \circ g) \circ g) \circ g) \ldots \)).   
\end{theorem}

\begin{proof}
	For simplicity, assume \(G\) is abelian and suppose \(G = \{g_1, g_2, \ldots, g_m\) and
	let \(g \in G\) be arbitrary. Note that
	\[
		g \circ g_i = g \circ g_j \implies g_i = g_j
	\]
	and so the set \(\{g \circ g_i : i \in \{1, m\}\}\) covers all elements of
	\(G\) exactly once. Therefore:
	\[
		g_1 \circ -g_m = g^m (g_1 \circ \ldots \circ g_m) \implies 1 = g^m.
	\]   
\end{proof}

\begin{corollary}[Fermat's Little Theorem]\label{flt}
	\(\forall\) prime \(p\), \(\text{gcd}(a, p) = 1 \implies a^{p - 1} \equiv 1 \mod p\)   
\end{corollary}

\begin{theorem}[Euler's Theorem]\label{EulerThm}
	For \(\Phi(N) = \big | \{ a | 1 \leq a \leq N, \text{gcd}(a, N) = 1 \} \big |\),
	\(| \mathbb{Z}_N^* | = \Phi(N)\) we have:
	\[
		\text{if gcd}(g, N) = 1 \implies g^{\Phi(N)} \equiv 1 \mod N
	\]  
\end{theorem}

\begin{corollary}
	For \(m = |G| > 1, e \in \mathbb{Z}, \text{gcd}(e, m) = 1\), define
	\(d = e^{-1} \mod m\) and function \(f_e \colon G \to G\) as \(f_e(g) = g^e\).    
	Then, \(f_e\) is a bijection whose inverse is \(f_d\).  
\end{corollary}

\begin{definition}[Cyclic Groups]
	A group, \(G\), is considered \textbf{cyclic} if \(\exists g \in G\) such
	that \(G = \{1 = g^0, g^1, g^2, \ldots, g^{m - 1}\}\). If this is the case,
	we say that \(g\) generates \(G\).   
\end{definition}

\begin{eg}
	For \(\mathbb{Z}_7^* = \{1, \ldots, 6\}\):
	\[
		\text{powers of 3: } \{ 3^0 \equiv 1, 3^1 \equiv 3, 3^2 \equiv 2, 3^3 \equiv 6, 3^4 \equiv 4, 3^5 \equiv 5\}
	\]
	\[
		\text{powers of 2: } \{ 2^0 \equiv 1, 2^1 \equiv 2, 2^2 \equiv 4, 2^3 \equiv 1, 2^4 \equiv 2, \ldots \}
	\]
	and so 3 is a generator of \(\mathbb{Z}_7^*\), but 2 is not.
\end{eg}

If there is no \(g \in G\) that generates \(G\), then \(G\) is not cyclic. Furthermore,
when \(g\) does not generate \(G\), it generates a \textbf{subgroup}:

\begin{definition}
	\(G^{\prime} \subseteq G\), is a subgroup if \((G^{\prime}, \circ)\) is a group.  
\end{definition}

\begin{theorem}[Lagrange's Theorem]
	If \(G^{\prime} \subseteq G\) is a subgroup, then \(|G^{\prime}|
	\text{ divides } |G|\).
\end{theorem}

\section{Fast Exponentiation}
Suppose we have an element \(g\) and we want to compute \(g^M\). Instead of
naively multiplying \(g\) \(M\) times, we can observe that if \(M = 2^m\):
\[
	g^{M} = g^{2^m} = g^{2^{m - 1}} \cdot g^{2^{m - 1}} = (g^{2^{m - 1}})^2.
\]
So for any \(M\) we can rewrite it as
\[
	M = \sum_{i = 0}^l m_i . 2^i,
\]
allowing us to calculate \(g^m\):
\[
	g^m = \prod_{i=0}^{l} g^{2^i},
\]
where each \(g^{2^i}\) can be calculated using the above trick. This
results in \(O(l^2)\) multiplications altogether if \(|M| = l\).

\begin{corollary}
	Fast exponentiation allows us to compute inverses efficiently because
	\(g^{-1} = g^{|G| - 1}\) since \(g^{|G|} = 1\).  
\end{corollary}

We've shown we can compute \(g^m\) efficiently given \(g\) and \(m\). But,
can we efficiently compute \(m\) from \(g^m\)? It's unknown, but conjectured
to be extremely difficult to do so.     