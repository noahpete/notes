\setcounter{chapter}{3}
\setcounter{lecture}{18}
\chapter{Number Theory and Public Key Cryptography}
\lecture{19}{20 Mar. 10:30}{Number Theory}
\section{Modular Arithmetic and Euclid's Algorithm}
We define the set of integers, \( \mathbb{Z} = \{ \ldots, -2, -1, 0, 1, 2, \ldots \} \), 
and natural numbers, \( \mathbb{N} = \mathbb{Z}^+ = \{ 1, 2, 3, \ldots \} \).

\begin{theorem}[Product of primes]\label{ProductOfPrimes}
	Every integer \( N > 1 \) can be written \emph{uniquely} as
	a product of (power of) primes.
\end{theorem}

\begin{lemma}[Division with remainder]\label{DivisionWithRemainder}
	Let \( a \in \mathbb{Z}, b \in \mathbb{Z}^+ \). $\exists$ unique
	integers $q, r$ such that \( a = q . b + r \) where \( 0 \leq r < b \),    
	and they can be efficiently computed in \emph{polynomial time}
	relative to the \emph{bit length}: i.e. \( \log_2 a + \log_2 b + O(1) \) 
\end{lemma}

With the ability to perform division in polynomial time, we are able
to find the \textbf{greatest common divisor} of two integers $a, b$:

\begin{definition}[Greatest common divisor]\label{gcd}
	Let \( a, b \in \mathbb{Z}^+ \). Then, there exists \( x, y \in 
	\mathbb{Z} \) such that \( \text{gcd}(a, b) = a.x  + b.y \).   
	Further, \( \text{gcd}(a, b) \) is the \emph{smallest} positive integer
	that can be written this way. 
\end{definition}

We claim there is an efficient algorithm, the \textbf{Extended Euclidean Algorithm},
that computes not only \( \text{gcd}(a, b) \), but also the \( x, y \) coefficients above:

\begin{algorithm}[H]\label{eea}
	\DontPrintSemicolon
	\caption{Extended Euclidean Algorithm}
	\KwData{\( a, b \in \mathbb{Z} \) where \( a \geq b > 0 \)  }
	\If{\(b|a\)}{
		\Return{\(x = 0, y = 1\) }
	}
	\Else{
		write \(a = q . b + r\) where \(0 < r < b\)  
	}
	\(x^{\prime}, y^{\prime} \gets \mathbf{ExtEuclid}(b, r) \)\; 
	\Return{\(x = y^{\prime}, y = x^{\prime} - y^{\prime} . q\) }\;
\end{algorithm}

\begin{proof}[Verification of Extended Euclidean Algorithm.]
	Assuming the recursive call is successful (by induction, we can), we will get back
	\(x^{\prime}, y^{\prime} \) that are the gcd of \(b\) and \(r\):   
	\[
		\begin{split}
			b . x^{\prime}  + r . y^{\prime} &= \text{gcd}(b, r) = \text{gcd}(a, b) \\
			b . x^{\prime} + (a - q . b) . y^{\prime} &= a . y^{\prime} + b (x^{\prime} - q . y^{\prime} ) \\
		\end{split}
	\]
	We note that \(y^{\prime} = x\) and \(x^{\prime} - q. y^{\prime} = y\) and so we are done.  
\end{proof}

We claim that the \hyperref[eea]{Extended Euclidean Algorithm} makes O(\(n\)) recursive calls.

\section{Group Theoretic View of Numbers}

\begin{definition}
	Let \(\mathbb{Z}_N = \{0, 1, 2, \ldots, N - 1 \}\), indicating the set of all possible 
	remainders of division by \(N\), further: \(\mathbb{Z}_N^* = \{ x \in \mathbb{Z}_N \colon \text{gcd}(x, N) = 1 \}\).
\end{definition}

\begin{remark}
	For example, \(\mathbb{Z}_6^* = \{1, 5\}\). Alternatively, \(\mathbb{Z}_7^* = \{
	1, 2, 3, 4, 5, 6 \}\), showing how \(\mathbb{Z}_p^* = \mathbb{Z}_p \setminus \{0\}\),
	where \(p\) is any prime number.   
\end{remark}
