\section{Encryption}
\lecture{5}{}{Encryption}
Our model of network security is based upon the following:
\begin{itemize}
	\item Two parties are attempting to communicate over a network
	\item Assume a powerful adversary (they have access to everything)
\end{itemize}

We define our goals:
\begin{itemize}
	\item Confidentiality
	\item Sender authenticity
	\item Message integrity
	\item Freshness
	\item Anonymity
\end{itemize}

Further, we must understand various techniques used to break encryption schemes.

\begin{definition}[Frequency Analysis]
	The process of counting the frequency of each letter in some cipher text to derive the plain text.
\end{definition}

\subsection{Symmetric Encryption}
Traditional cryptography utilizes a single key for both "locking" and "unlocking" the message. This is called \textbf{symmetric encryption}, which is utilized by AES.

\begin{definition}[Symmetric Encryption]
	An encryption scheme in which both sides require the same key to encrypt and decrypt the message.
\end{definition}
\begin{remark}
	Symmetric encryption schemes are always going to be \textbf{reversible}.
\end{remark}

\begin{definition}[Advanced Encryption Standard]
	Advanced Encryption Standard (AES) is widely believed to be secure, uses 128-bit block size, and a variable key size (128 or 256 bits). It encompasses ten rounds of substitution/permutation operations, each time with a different subkey of the original key.
\end{definition}

Symmetric encryption is generally \emph{fast} and it provides \emph{confidentiality}. With the use of message authentication code (MACs), symmetric encryption schemes can also provide \emph{message integrity}.

\subsection{Asymmetric Encryption}
The issue with symmetric encryption schemes is that \emph{key distribution is the weak link}. Not only is it hard to revoke key access once permitted, but consider applying the schemes to the web. This quickly becomes infeasible. To help, we can utilize \textbf{asymmetric encryption}.

\begin{definition}[Asymmetric Encryption]
	An encryption scheme in which each party has a \textbf{public} key as well as a \textbf{private} key. Only a party's private key is able to unlock something locked with their corresponding public key.
\end{definition}

Asymmetric encryption, or public key cryptography, provides \emph{confidentiality} and \emph{sender authenticity}. However, we can use asymmetric encryption to \emph{establish} a \textbf{shared secret key} between two parties. Then we can use \emph{symmetric} encryption with that shared key.

\subsection{Public Key Infrastructure}
The Public Key Infrastructure (PKI) distributes public keys safely, via certificates. When a client wants to transmit something securely to a server, it may then request the server's public key and use that to encrypt their message. \par

Public keys for big Certificate Authorities (CAs) are built into browsers. Different certification "strengths" exist depending on the level of identity verification.

\begin{figure}[H]
	\centering
	\incfig[0.4]{asymsymkeyex}
	\caption{Client-server interaction during key exchange.}
	\label{fig:asymsymkeyex}
\end{figure}

\subsection{TLS/SSL}
\textbf{Transport Layer Security / Secure Sockets Layer}, commonly exercised via https://, is responsible for the encryption of all content that goes into TCP payload. \textbf{SSL} is usually implemented by the server, and then the server decrypts HTTPS traffic before proxying dynamic page requests to the backend.

\subsection{Hash Functions and Passwords}
It's a bad idea to store passwords as vanilla text in a database. It's a far better idea if the server hashes password using a one-way hash function. \par

Currently, \textbf{512 bit SHA-2} is the standard hash function used since it is resistant to collision attacks. This way, the server can store hashed passwords and then simply compare a hashed user input to the value stored in the database to see if there is a password match. \par

\textbf{Rainbow tables} speed up \emph{brute force} attacks with pre-computed tables. If you don't use a \textbf{salt}, then you're susceptible to these attacks.

\begin{definition}[Salt]
	A random number appended to the password plain text.
\end{definition}