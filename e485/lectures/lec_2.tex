\lecture{2}{}{Static Pages}
\subsection{Document Object Model (DOM)}
HTML tags form a tree called the \textbf{document object model (DOM)}: 

\begin{lstlisting}[language=html]
<html>
	<head></head>
	<body>
	...
	</body>
</html>
\end{lstlisting}

The DOM is a data structure built from the HTML. In the DOM, everything is a \textbf{node}. All HTML elements are element nodes, with text inside HTML being text nodes.

\subsection{HTML5}
HTML5 merges the advancements made during HTML's lifespan, specifically those from XHTML and browser-specific extensions of HTML.

\section{URLs and HTTP}
We can download a webpage using \textbf{curl}:
\begin{lstlisting}[language=bash]
$ curl --verbose http://cse.eecs.umich.edu/ > index.html
\end{lstlisting}
or more generally:
\begin{lstlisting}[language=bash]
$ curl --verbose protocol://server:port/path?query#fragment > out.html
\end{lstlisting}

\subsection{URLs}
We can observe each part of the URL in more detail:

\begin{itemize}
	\item \texttt{protocol} - tells the server what protocol to use, i.e. what "language" to speak
	\item \texttt{server} - locates the machine we want to talk to
	\begin{itemize}
		\item DNS lookup translates server name into an IP address (e.g. localhost and 127.0.0.1)
	\end{itemize}
	\item \texttt{port} - used to identify a specific service
	\begin{itemize}
		\item we can check what ports are open using \texttt{\$ nmap cse.eecs.umich.edu}
	\end{itemize}
	\item \texttt{path} - a file name relative to the server root
	\begin{itemize}
		\item default is \texttt{/index.html}
	\end{itemize}
	\item \texttt{query} - a general-purpose string of parameters that the server can use
	\item \texttt{fragment} - identified at the client, ignored by server; think page navigation
\end{itemize}

\subsection{HTTP}
Hypertext Transfer Protocol essentially reflects the request/response protocol outlined prior with the additional constraints that a \textbf{server cannot open connection to the client} and the protocol is completely \textbf{stateless}.

\section{Requests}

\begin{definition}[\texttt{GET} Request]
	A \texttt{GET} request is simply the prompting of a server for data. Requests using \texttt{GET} should only retrieve data.
\end{definition}

\subsection{General Request-Response Structure}

So far we've seen what a \texttt{GET} request looks like. Now we will look at the other two types of requests: \texttt{HEAD} and \texttt{POST}. Take the following example server request-response:
\begin{lstlisting}[language=bash]
$ curl --verbose http://cse.eecs.umich.edu/ > index.html
* Connected to cse.eecs.umich.edu	
(141.212.113.143) port 80 (#0)
> GET / HTTP/1.1
> Host: cse.eecs.umich.edu
> User-Agent: curl/7.54.0
> Accept: */*
\end{lstlisting}

The above request to \texttt{cse.eecs.umich.edu} contains the following headers:
\begin{itemize}
	\item \texttt{Host}: distinguishes between DNS names sharing a single IP address
	\item \texttt{User-Agent}: which browser is making the request
	\item \texttt{Accept}: which content (i.e. file) types the client will accept
\end{itemize}

The response from the server will start with a status code:
\begin{itemize}
	\item 1XX: informational
	\item 2XX: successful
	\item 3XX: redirection error
	\item 4XX: server error
\end{itemize}
Further, the response will have accompanying headers, most of which are optional. Take the following example:
\begin{lstlisting}[language=bash]
$ curl --verbose http://cse.eecs.umich.edu/
* Connected to cse.eecs.umich.edu
> GET / HTTP/1.1
...
< HTTP/1.1 200 OK
< Date: Tue, 12 Sep 2017 20:04:20 GMT
< Server: Apache/2.2.15 (Red Hat)
< Accept-Ranges: bytes
< Connection: close
< Transfer-Encoding: chunked
< Content-Type: text/html; charset=UTF-8	
\end{lstlisting}

Where the \texttt{Content-Type} describes the "file" type and encoding (e.g. \texttt{text/html} or \texttt{image/png}). The type and encoding are known as the \textbf{MIME type}. Next is the character encoding, which is often just \texttt{UTF-8}, which corresponds to Unicode, which covers almost all of the characters and symbols in the world. It shares the first 128 values with ASCII, although unlike ASCII, Unicode is expandable.

\subsection{\texttt{HEAD} Requests}
\texttt{HEAD} requests are useful for checking if a page has changed or for caching stuff that doesn't change much and updating it. Along with other typical headers, it returns a \texttt{Last-Modified} header that may look like the following:
\begin{lstlisting}[language=bash]
< Last-Modified: Thu, 28 May 2015 13:40:55 GMT	
\end{lstlisting}

\begin{definition}[\texttt{HEAD} Request]
	A \texttt{HEAD} request acts just like a \texttt{GET} request, but doesn't return the body, only the headers.
\end{definition}

\subsection{\texttt{POST} Request}
\begin{definition}[\texttt{POST} Request]
	A \texttt{POST} request sends data from the client to the server.
\end{definition}

\section{HTTP/1.0 vs HTTP/1.1 vs HTTP/2}
The largest difference between HTTP/1.0 and HTTP/1.1 is the number of transmission control protocols that can occur simultaneously. Take the following HTML, for example:
\begin{lstlisting}[language=HTML]
<html>
  <body>
    <p>Block M<p>
    <img src="http://cse.eecs.umich.edu/eecs/images/image1.png">
    <img src="http://cse.eecs.umich.edu/eecs/images/image2.png">
  </body>
</html>	
\end{lstlisting}

In the case of:
\begin{itemize}
	\item HTTP/1.0: we will need to (1) set up a connection to load HTML, (2) set up another connection and load image1.png, and then (3) set up another connection to load image2.png; \emph{this uses 3 connections}
	\item HTTP/1.1: we can use one connection to load the HTML, then load image1, then load image2 in a sequence; \emph{this uses 1 connection} 
\end{itemize}

HTTP/2 has the same methods, status codes, etc. as HTTP/1.1. One new feature is the idea of \textbf{server push}, where the server supplies data it knows a web browser will likely need to render a web page \emph{without waiting} for the browser to examine the first response.