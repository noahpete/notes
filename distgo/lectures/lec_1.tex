\chapter{Let's Go}
\section{Go: An Overview}
\newtheorem{file}{File}

It's first important to note the advantages of Go, and to see why we are using it to build distributed systems. Go is a particularly good language for distributed systems because it is one that can simultaneously:
\begin{enumerate}[label=(\roman*)]
	\item compile annd ru programs faster than an interpreted language like Ruby or Python;
	\item write highly concurrent (parallel) programs;
	\item run directly on underlying hardware; and
	\item use modern features like packages
\end{enumerate}

\section{JSON over HTTP}
We begin our construction of distributed systems by building a commit log JSON over HTTP service. This is a widely used API, so it is a practical example at a small scale. For reference, a \textbf{commit log} is a data structure for an append-only sequence of records, ordered by time.


\begin{file}[internal/service/log.go] \:
	\begin{lstlisting}[language=Go]
package server

import (
	"fmt"
	"sync"
)
	
type Log struct {
	mu sync.Mutex
	records []Record
}

func NewLog() *Log {
	return &Log{}
}

func (c *Log) Append(record Record) (uint64, error) {
	c.mu.Lock()
	defer c.mu.Unlock()
	record.Offset = uint64(len(c.records))
	c.records = append(c.records, record)
	return record.Offset, nil
}

func (c *Log) Read(offset uint64) (Record, error) {
	c.mu.Lock()
	defer c.mu.Unlock()
	if offset >= uint64(len(c.records)) {
		return Record{}, ErrOffsetNotFound
	}
	return c.records[offset], nil
}

type Record struct {
	Value []byte `json:"value"`
	Offset uint64 `json:"offset"`
}

var ErrOffsetNotFound = fmt.Errorf("offset not found")
	\end{lstlisting}
\end{file}

